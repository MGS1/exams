% !TEX TS-program = pdflatex
% !TEX encoding = UTF-8 Unicode

\documentclass[12pt,addpoints]{exam} % use larger type; default would be 10pt
\usepackage[utf8]{inputenc} % set input encoding (not needed with XeLaTeX)
\usepackage{listings}
\usepackage{amssymb} % for \nmid


%%% Examples of Article customizations
% These packages are optional, depending whether you want the features they provide.
% See the LaTeX Companion or other references for full information.

%%% PAGE DIMENSIONS
\usepackage{geometry} % to change the page dimensions
\geometry{a4paper} % or letterpaper (US) or a5paper or....
% \geometry{margins=2in} % for example, change the margins to 2 inches all round
% \geometry{landscape} % set up the page for landscape
%   read geometry.pdf for detailed page layout information


% \usepackage[parfill]{parskip} % Activate to begin paragraphs with an empty line rather than an indent

%%% PACKAGES
\usepackage{booktabs} % for much better looking tables
\usepackage{array} % for better arrays (eg matrices) in maths
\usepackage{paralist} % very flexible & customisable lists (eg. enumerate/itemize, etc.)
\usepackage{verbatim} % adds environment for commenting out blocks of text & for better verbatim
\usepackage{subfig} % make it possible to include more than one captioned figure/table in a single float
\usepackage{amsthm} % for proof environment
% These packages are all incorporated in the memoir class to one degree or another...
\usepackage{amsfonts} % allows use of mathbb 
\usepackage{amsmath}
\usepackage{multicol}
\usepackage{graphicx} % support the \includegraphics command and options


%%% HEADERS & FOOTERS
%\usepackage{fancyhdr} % This should be set AFTER setting up the page geometry
%\pagestyle{fancy} % options: empty , plain , fancy
%\renewcommand{\headrulewidth}{0pt} % customise the layout...
%\lhead{}\chead{}\rhead{}
%\lfoot{}\cfoot{\thepage}\rfoot{}

%%% SECTION TITLE APPEARANCE
\usepackage{sectsty}
\allsectionsfont{\sffamily\mdseries\upshape} % (See the fntguide.pdf for font help)
% (This matches ConTeXt defaults)

%%% ToC (table of contents) APPEARANCE
\usepackage[nottoc,notlof,notlot]{tocbibind} % Put the bibliography in the ToC
\usepackage[titles,subfigure]{tocloft} % Alter the style of the Table of Contents
\renewcommand{\cftsecfont}{\rmfamily\mdseries\upshape}
\renewcommand{\cftsecpagefont}{\rmfamily\mdseries\upshape} % No bold!
\newcommand{\Dgrad}[1]{\textsf{grad}\,#1\;}
\newcommand{\del}[1] {\mathrm{d}#1\ }
\newcommand{\Q}{\mathbb{Q}}
\newcommand{\F}{\mathbb{F}}
\newcommand{\C}{\mathbb{C}}
\newcommand{\Z}{\mathbb{Z}}
\newcommand{\R}{\mathbb{R}}
\newcommand{\N}{\mathbb{N}}
\newcommand{\PP}{\mathbb{P}}
\newcommand{\arcsec}{\mbox{arcsec}}
\newcommand{\arccsc}{\mbox{arccsc}}
\newcommand{\arccot}{\mbox{arccot}}
\newcommand{\csch}{\mbox{csch}}
\newcommand{\sech}{\mbox{sech}}

%%% NEW ENVIRONMENTS
\newtheorem{theorem}{Theorem}
\newtheorem{definition}[theorem]{Definition}
\newtheorem{example}[theorem]{Example}
\newtheorem{proposition}[theorem]{Proposition}
\newtheorem{corollary}[theorem]{Corollary}
\newtheorem{lemma}[theorem]{Lemma}

%%% END Article customizations

%%% The "real" document content comes below...


\usepackage{color}
\lstset{numbers=left}
\begin{document}

\printanswers

\title{575: Number Theory}
\maketitle



\begin{center}
    This exam has \numquestions\ questions for a total of \numpoints\
    points.You have {\bf 12 Hours} to complete this exam.
\end{center}

% 35 lectures, 12 hours, so 20.5ish points per lecture.

\begin{questions}

    % Lecture 1 part I
    \question[5] Prove that, for all positive integers $n$,
        $$1^3 + 2^3 + \cdots + n^3 = \frac{n^2(n+1)^2}{4}.$$
    \begin{solution}
        Base case: $1^3 = \frac{1^2(2)^2}{4} = 1$.

        Induction Step: 
        $$\sum_{k=1}^{n+1} k^3 = (n+1)^3 + \sum_{k=1}^n k^3$$
        $$  (n+1)^3 + \frac{n^2(n+1)^2}{4}
          = \frac{ 4(n+1)^3 + n^2(n+1)^2 }{4}$$
        $$ = \frac{ (n+1)^2 ( 4(n+1) + n^2 ) }{4}
           = \frac{ (n+1)^2 (n+2)^2 }{4}.$$        
    \end{solution}


    % Lecture 1 part II
    \question
    \begin{parts}
        % Niven/et. al. 1.2.9
        \part[3] Show that if $ac | bc$, then $a | b$.
        \begin{solution}
            The hypothesis implies that there is $m$ such that,
            $$bc = acm$$
            $$b = am$$
            i.e., $a | b$.
        \end{solution}
        % Niven/et. al. 1.2.10
        \part[3] Show that if $a | b$ and $c | d$ then $ac | bd$.
        \begin{solution}
            By hypothesis there is $m$ and $n$ such that
            $$b = am,\ d = cn$$
            $$bd = ac(mn)$$
            So $ac | bd.$
        \end{solution}
    \end{parts}

    % Lecture 1 part III
    \question (\totalpoints\ points)
    \begin{parts}
        \part[1] State the division algorithm
        \begin{solution}
            Given any integers $a$ and $b$, with $a > 0$, there exist
            unique integers $q$ and $r$ such that $b = qa + r$, $0 \le
            r < a.$
        \end{solution}
        % Niven/et. al. 1.2.11
        \part[8] Prove that $4 \nmid (n^2 + 2)$ for any integer $n$.
        \begin{solution}
            Using the division algorithm write $n = 4q + r$ where $0 \le r
            < 4.$ Then 
            $$n^2 + 2 = 16q^2 + 8qr + r^2 + 2$$
            $$ = 4(4q^2 + 2qr) + r^2 + 2.$$
            Examining possibilities for $r$ we see that the possibilities
            for $r^2 + 2$ are: 2, 3, 6 and 11. If $n^2 + 2$ were divisible
            by 4, then by the linear combination rule of divisbility, we'd
            have that $n^2 +2 - 4(4q^2 + 2qr) = r^2 + 2$ were divisible by
            4. However, by exhaustion 2, 3, 6, and 11 are not. 
        \end{solution}       
    \end{parts}      
\end{questions}

\end{document}
