% !TEX TS-program = pdflatex
% !TEX encoding = UTF-8 Unicode

\documentclass[12pt,addpoints]{exam} % use larger type; default would be 10pt
\usepackage[utf8]{inputenc} % set input encoding (not needed with XeLaTeX)
\usepackage{listings}
\usepackage{amssymb} % for \nmid


%%% Examples of Article customizations
% These packages are optional, depending whether you want the features they provide.
% See the LaTeX Companion or other references for full information.

%%% PAGE DIMENSIONS
\usepackage{geometry} % to change the page dimensions
\geometry{a4paper} % or letterpaper (US) or a5paper or....
% \geometry{margins=2in} % for example, change the margins to 2 inches all round
% \geometry{landscape} % set up the page for landscape
%   read geometry.pdf for detailed page layout information


% \usepackage[parfill]{parskip} % Activate to begin paragraphs with an empty line rather than an indent

%%% PACKAGES
\usepackage{booktabs} % for much better looking tables
\usepackage{array} % for better arrays (eg matrices) in maths
\usepackage{paralist} % very flexible & customisable lists (eg. enumerate/itemize, etc.)
\usepackage{verbatim} % adds environment for commenting out blocks of text & for better verbatim
\usepackage{subfig} % make it possible to include more than one captioned figure/table in a single float
\usepackage{amsthm} % for proof environment
% These packages are all incorporated in the memoir class to one degree or another...
\usepackage{amsfonts} % allows use of mathbb 
\usepackage{amsmath}
\usepackage{multicol}
\usepackage{graphicx} % support the \includegraphics command and options


%%% HEADERS & FOOTERS
%\usepackage{fancyhdr} % This should be set AFTER setting up the page geometry
%\pagestyle{fancy} % options: empty , plain , fancy
%\renewcommand{\headrulewidth}{0pt} % customise the layout...
%\lhead{}\chead{}\rhead{}
%\lfoot{}\cfoot{\thepage}\rfoot{}

%%% SECTION TITLE APPEARANCE
\usepackage{sectsty}
\allsectionsfont{\sffamily\mdseries\upshape} % (See the fntguide.pdf for font help)
% (This matches ConTeXt defaults)

%%% ToC (table of contents) APPEARANCE
\usepackage[nottoc,notlof,notlot]{tocbibind} % Put the bibliography in the ToC
\usepackage[titles,subfigure]{tocloft} % Alter the style of the Table of Contents
\renewcommand{\cftsecfont}{\rmfamily\mdseries\upshape}
\renewcommand{\cftsecpagefont}{\rmfamily\mdseries\upshape} % No bold!
\newcommand{\Dgrad}[1]{\textsf{grad}\,#1\;}
\newcommand{\del}[1] {\mathrm{d}#1\ }
\newcommand{\Q}{\mathbb{Q}}
\newcommand{\F}{\mathbb{F}}
\newcommand{\C}{\mathbb{C}}
\newcommand{\Z}{\mathbb{Z}}
\newcommand{\R}{\mathbb{R}}
\newcommand{\N}{\mathbb{N}}
\newcommand{\PP}{\mathbb{P}}
\newcommand{\arcsec}{\mbox{arcsec}}
\newcommand{\arccsc}{\mbox{arccsc}}
\newcommand{\arccot}{\mbox{arccot}}
\newcommand{\csch}{\mbox{csch}}
\newcommand{\sech}{\mbox{sech}}

%%% NEW ENVIRONMENTS
\newtheorem{theorem}{Theorem}
\newtheorem{definition}[theorem]{Definition}
\newtheorem{example}[theorem]{Example}
\newtheorem{proposition}[theorem]{Proposition}
\newtheorem{corollary}[theorem]{Corollary}
\newtheorem{lemma}[theorem]{Lemma}

%%% END Article customizations

%%% The "real" document content comes below...


\usepackage{color}
\lstset{numbers=left}
\begin{document}

\printanswers

\title{Econ 105}
\maketitle



\begin{center}
    This exam has \numquestions\ questions for a total of \numpoints\
    points. You have {\bf 2 Hours} to complete this exam.
\end{center}

% Syllabus
%
% Microeconomics:
% -------------
%
% 1. scarcity, effiency, specialization
% 2. Supply and demand: 1 MC, 1 SA, 7 points 
% 3. Consumer demand: 1 MC,  1 SA, 7 points
% 4. Supply: production, costs and profits: 3 MC, 6 points  
% 5. Perfect competition 1 MC, 1 SA, 12 points
% 6. Imperfect competition: 1 MC, 1 SA, 11 points
% 7. Factor prices, incomes, and income distribution: 3 MC, 6 points
% 8. market efficiency and market failures: 3 MC, 6 points
%
% Macroeconomics
% -------------
%
% 9. Macro overview
% 10. Measuring total output and inflation: 1 MC, 1SA, 13 points
% 11. Short-run AE model of output: 1 MC, 1SA, 19 points
% 12. Fiscal policy and government budgets: some of the %11 is here.
%               464-466, 624-632, 636-639
% 13. Money, interest rate, and monetary policy
%               117-119, chp 26, 503-519
% 14. AD and AS
%               442-447, 543-548, 391-398, 520-521, 454-455
\begin{questions}

  \section{Multiple Choice}

  % 2
  \question[2] At the equilibrium price for a good:
  \begin{choices}
    \choice prices will remain unchanged, even if there is excess
    demand.
    \choice there may be excess demand for the good but not excess
    supply.
    \choice shifts in the supply or demand curves will not cause price
    changes.
    \CorrectChoice the quantity supplied of the good must equal the
    quantity demanded.
  \end{choices}

  % 3
  \question[2] A consumer will be maximizing utility at
  \begin{choices}
    \choice any intersection of the budget line and an indifference
    curve
    \choice any point on the highest indifference curve shown on the
    indifference diagram.
    \CorrectChoice the point where the indifference curve is tangent
    to the budget line
    \choice any point on the budget line
    \choice any point inside the budget line.
  \end{choices}

  % 4
  \question[2] The price of coal rose and the quantity sold
  fell. Which of the following is consistent with this observation
  (everything else being equal)?
  \begin{choices}
    \choice the price of oil rose
    \choice the productivity of coal miners rose
    \CorrectChoice more costly coal mining equipment was installed
    \choice the population increased
  \end{choices}

  % 4
  \question[2] The law of increasing (opportunity) costs explains why
  the production possibilities frontier  (PPF) is
  \begin{choices}
    \choice downward sloping
    \choice upward sloping
    \choice bowed inward
    \CorrectChoice bowed outward
    \choice undefined because no market will exist in this case.
  \end{choices}

  % 4
  \question[2] Marginal cost is the
  \begin{choices}
    \choice total cost divided by the number of units produced.
    \choice variable cost divided by the number of units produced
    \choice extra average cost of producing an additional unit of
    output
    \CorrectChoice extra cost of producing an additional unit of
    output
  \end{choices}

  % 5
  \question[2] Which of the following is not characteristic of a
  perfectly competitive industry?
  \begin{choices}
    \CorrectChoice Price is the primary decision variable of
    individual firms in the industry
    \choice There are many buyers and sellers of the industry product
    \choice There are no barriers to exit from or entry to the
    industry
    \choice The products of all firms in the industry are identical
    \choice None of the above.
  \end{choices}

  %7
  \question[2] Roughtly speaking, disposable income is equal to
  national income
  \begin{choices}
    \choice minus taxes and transfers.
    \choice plus taxes and transfers.
    \choice minus taxes.
    \choice plus transfers.
    \CorrectChoice minus taxes but plus transfers.
  \end{choices}

  % 6
  \question[2] Market power referes to the ability of a consumer or
  producer to
  \begin{choices}
    \choice influence the equilibrium quantity in the market.
    \choice influence entry and exit in the market.
    \CorrectChoice influence the equilibrium price in the market.
    \choice engage in non-price competition.
  \end{choices}

  % 11
  \question[2] The Marginal propensity to consume (MPC) is
  \begin{choices}
    \choice consumption divided by disposable income.
    \choice consumption divided by GDP.
    \CorrectChoice the change in consumption divided by the change in
    disposable income.
    \choice the change in consumption divided by the change in GDP.
  \end{choices}

  %7
  \question[2] Over the last few decades, rising hourly wages in the
  US have been accompanied by a decline in the number of hours worked
  each week. This suggests that in the US labor market,
  \begin{choices}
    \CorrectChoice the income effect has been dominant.
    \choice the substitution effect does not exist at all.
    \choice the US worker is no longer productive.
    \choice workers have become increasingly lazy.
  \end{choices}

  % ???
  \question[2] If you were interested in estimating the potential
  increase in consumer demand for new automobiles next year, the
  number you would be most interested in forecasting would be
  \begin{choices}
    \choice net domestic product
    \choice national income
    \CorrectChoice disposable income
    \choice gross domestic product
    \choice gross national product
  \end{choices}

  % 8
  \question[2] Pure public good lead to market failure because
  \begin{choices}
    \choice these are goods most likely to give rise to
    non-competetive behavior by firms
    \choice these are goods that can be more cheaply produced by the
    public sector.
    \CorrectChoice it is not possible to confine the benefits of the
    goods to those who pay for them
    \choice private firms are prohibited from producing them, though
    they could do so more cheaply than the public sector.
  \end{choices}
 
  % 7
  \question[2] When factor markets are competitive, it always pays a
  profit-maximizing firm to
  \begin{choices}
    \choice use more of the factor.
    \choice bid very low prices for factors.
    \choice reduce the use of all factors.
    \CorrectChoice use that quantity of a factor that makes MRP equal
    to the price of the factor.
  \end{choices}

  
  % 10
  \question[2] Which of the following would not be included in gross
  investment?
  \begin{choices}
    \choice the total value of newly-built residential housing
    \CorrectChoice Ms. Smith's purchase of 100 shares of Xerox stock.
    \choice the purchase of a new truck by U-Haul to replace a
    worn-out truck.
    \choice the purchase of new machine tools by Ford Motor Company.
    \choice an increase in Campbell Soup's inventory of canned tomato
    soup.
  \end{choices}
  
  % 8 
  \question[2] Negative externalities imply
  \begin{choices}
    \choice the marginal social cost exceeds the marginal private
    cost.
    \choice the private market supplies too much of the commodity.
    \choice private individuals ignore the costs their actions impose
    on others.
    \choice taxes on activities generating negative externalities can
    reduce the inefficiency.
    \CorrectChoice all of the above.
  \end{choices}

  % 8
  \question[2] Allocative efficiency describes an economy
  \begin{choices}
    \choice which has attained an appropriate distribution of income.
    \choice in which marginal revenue equals marginal cost.
    \CorrectChoice where no one can be made better off without making
    someone else worse off.
    \choice in which there will be no need for a government.
    \choice in which markets will not exist because they are not
    necessary when allocative efficiency prevails.
  \end{choices}

  \section{Short Answer}

  % 3 
  \question[5] Economists give two reasons why most demand curves have
  a negative slope. Name and explain one of them.
  \begin{solution}
    There are the income and substition effects:

    As the price of a commodity falls the good becomes relatively
    cheaper compared to substitute commodities and thus become
    prefered so demand rises. This is called the substition effect.

    Also as the price falls the real buying power of a consumer rises,
    thus enabling the consumer to buy more of that good so demand
    rises. This is called the income effect
  \end{solution}

  % 2
  \question[5] How would you expect the price elasticity of Coca-Cola
  to compare to the price elasticity of soft drinks as a whole?

  \begin{solution}
    The price elasticity of a good is the percent change in quanity
    consumed over the percent change in price. Since Coke is one of
    many soft drink options, the price elasticity of demand for
    Coca-Cola is more negative than for soft drinks as a whole. As the
    price increases, the quantity demand drops off more because there
    are many substititues. If the price increased for soft drinks as a
    whole, there are fewer subsititues (hard drinks, coffee, etc.)
    which are not as complete substititues.
  \end{solution}

  % 5 ?
  \question (\totalpoints\ points) A few years ago a report aired on
  60 minutes which showed the unhealthy conditions in the American
  chicken industry, and suggested that consumers were risking
  salmonella infection each time they ate chicken.

  Using graphs and words answer the following questions about the
  impact of this report on the chicken industry and on the individual
  chicken grower. You should assume that the US industry producing
  chickens is a perfectly competitive industry, and had been in a
  state of long run equilibrium prior to the report.

  \begin{parts}
    
    \part[5] Explain what will happen to the industry demand curve
    and the equilibrium levels of price and output in the industry in
    the short run. What will happen to the profits earned by chicken
    growers in the short run?

    \begin{solution}
      In the short run the demand curve shifts left, meaning that at
      the current price point more chickens will be produced than
      bought. There will be a surplus of goods and profits will be
      negative. 
    \end{solution}

    \part[5] Explain what will happen to the number of chicken
    growers in the industry, the equilibrium levels of price and
    output in the industry, and the profits earned by chicken growers
    in the long run.

    \begin{solution}
      As Growers lose profits they will leave and the supply curve
      also shifts left. This causes prices to rise in the
      industry. Thus individual firms see profits rise from less than
      zero to 0 in the long-run. Industry price remains to close to
      the original price in the long run, but industry output falls
      even further than in short run due to exit from industry.
    \end{solution}

  \end{parts}

  % 6
  \question[9] In general, economists dislike monopolies because, from
  a social point of view, they are inefficient. Explain the basis for
  their dislike (ie., explain why a profit-maximizing monopoly is
  inefficient), using a clearly labelled graph to illustrate your
  answer.


  \begin{solution}

    For any firm the point where profit will be maximized is when
    $MC=MR$. This is because if $MC < MR$ then one additional unit
    will bring in more revenue than it costs and therefore it would be
    productive to sell more and vice versa.

    \includegraphics{monopoly-deadweight-profit.jpg}

    For a monopoly the slope of the MR curve is 2 times the slop of
    the AD curve. At $A = (Q2, P1)$ the $MC = P1$ and therefore the
    price is an accurate reflection of the value of that good.  So a
    monopoly will produce at $QM$ which is where $MC = MR$ but charge
    $P^*$ which is the highest price they can charge. At this point
    $P^* > MC$ so the price signals are no longer accurate reflection
    of the goods value to society. Therefore consumers are actually
    purchasing less than they normally would.

    In perfect competition the quantity producted ($Q2$) is produced
    at price $P1$ and sold at that price. Which means that more is
    produced than in a monopoly, and more is produced for
    less. Obviously this is better for all. 

    Plus a monopoly eanrs economic profits which people find
    harmful. In the longrun in perfect competition profits are zero.
    
  \end{solution}

  \question (\totalpoints\ points) 
  \begin{parts}
    % 10
    \part[4] Define nominal GDP. Be complete but concise.
    \begin{solution}
      Nominal GDP is a measure of a nations final output in current
      dollars. This includes depreciation. It is the sum of
      Consumption, Gross Investment, Government spending (not
      transfers), and net exports.
    \end{solution}
    \part[7] Between 1979 and 1994 nominal GDP rose in the US from
    2489 billion dollars to 6738 billion. Based on these figures,
    would you say that Americans living in 1994 were almost three
    times as well off as they were in 1979? Explain.
    \begin{solution}
      No for these reasons:
      \begin{itemize}
      \item Inflation makes camparison of nominal GDP meaningless. How
        much did a car cost in 1979 compared to now? What if bread was
        1 dollar in 1979 but 5 dollars in 1994?
      \item The population as grown. Three dollars among 3 people is
        better than 30 among 100.
      \item Income distribution: How much is concentrated into one
        portion of the economy.
      \item Other intangibles that are not measured by dollars such as
        discrimination, safety, technological advances, externalities.
      \end{itemize}
    \end{solution}
  \end{parts}

  % 11 & 12
  \question (\totalpoints\ points) Provide complete but concise
  answers to each question below. Be sure to show all calculations
  used to arrive at numerical answers.

  Consider the following hypothetical scenario. An economist in Japan
  has been collecting data (measured in yen) on the behavior of the
  country's citizens and has found:

  \begin{center}
    \begin{tabular}{ccccccccc}
      $Y$ & $T-R$ & $DI$ & $C$ & $I^*$ & $G$ & $X$ & $M$ & $AE$ \\
      \hline
      100 & 50 & 50   &  85  & 40 & 20 & 20 & 20 & 145 \\
      130 & 50 &        & 106 & 40 & 20 & 20 & 26 & 160 \\
      160 & 50 & 110 & 127 & 40 & 20 & 20 & 32 & 175 \\
      190 & 50 & 140 &        & 40 & 20 & 20 &      & 190 \\
      210 & 50 & 160 & 162 & 40 & 20 & 20 & 42 &        \\
    \end{tabular}
  \end{center}

  \begin{parts}
    \part[2] What are the values of DI and AE that are missing in the chart?
    \begin{solution}
      $$DI = 80$$
      $$AE = 200$$
    \end{solution}

    \part[2] What are the MPC and the MPS?
    \begin{solution}
      $$MPC = 0.7$$
      $$MPS = 0.3$$
    \end{solution}
 
    \part[1] What is the consumption function?
    \begin{solution}
      $$C(DI) = 50 + 0.7DI\ yen$$
    \end{solution}

    \part[2] How do imports relate to $Y$?
    \begin{solution}
      $$M = 0.2Y$$
    \end{solution}

    \part[2] The equilibrium level of $Y$ is 
    \begin{solution}
      $$190$$
    \end{solution}
      
    \part[4] What effect do imports have on the the multiplier for
    government spending? Explain carefully.
    \begin{solution}
      Imports depend positively on $Y$. They are therefore a source
      of leakage in the model; whenever $Y$ increases, imports rise
      as well. This increase in imports subtracts from the demand
      for domestic goods and services at every stage in the
      multiplier process, resulting in a smaller net increase in $Y$
      for a given initial increase in $G$. In other words, the
      existence of imports that vary positively with the level of
      $Y$ lowers the multiplier for government spending (as well as
      the multipliers for exports, investment, and net taxes). 
    \end{solution}
    
    \part[2] In this economy, which includes the government and
    foreign trade sectors, the government spending multipler is?
    \begin{solution}
      $$2$$
    \end{solution}

    \part[4] If, households purchase more foreign goods such as
    Lincoln Continentals what would you expect to happen to the level
    of equilibrium GDP in Japan? Explain.
    \begin{solution}
      Since $GDP = C + I^* + G + X - M$ if more is spent on $M$ then
      GDP will fall since net imports will fall so the change in Net
      exports is less than zero and therefore $\Delta AE < 0$ and at
      equilibrium AE = y (there's a recursive progression here but in
      the end we can just take the limit).
    \end{solution}
    
    \part[2] If exports doubled, what would happen to the level of
    equilibrium $Y$?
    \begin{solution}
      If exports rose by $20$ dollars then equilibrium $Y$ would rise
      by (multiplier)(change) = 40.
    \end{solution}

    \part[4] What is the budgest balance equal to in this model? Does
    it vary with $Y$? Is this realistic? Why or why not?
    \begin{solution}
      The budget balance is $T - R - G = 30$. The budget balance does
      not vary with $Y$. This is not realistic, as in reality tax
      receipts decline and transfers rise during economic recessions
      and vice versa. Therefore, we would expect the budget balance to
      decline when $Y$ falls and rise when $Y$ increases. 
    \end{solution}

    \part Suppose that the government in this economy wants to
    increase equilibrium $Y$ by 40 dollars by using fiscal policy.
    \begin{subparts}
      \subpart[2] First, suppose it decides to achieve its goal by
      changing the level of government spending. What should it do?
      \begin{solution}
        It should increase $G$ by 20.
      \end{solution}
      \subpart[2] Suppose it decides to achieve its goal by changing
      the net lump-sum taxes. What should it do?
      \begin{solution}
        It should decrease $T - R$ by 28.6.
      \end{solution}
      \subpart[1] In both these ways, will the budget balance
      increase, decrease or remain unchanged?
      \begin{solution}
        decrease
      \end{solution}
    \end{subparts}
  \end{parts}

  \question (\totalpoints\ points) The First Baltimore Bank is a
  member of the banking system and faces a reserve requirement of
  20\%. Its balance sheet is:

  \begin{center}
    \begin{tabular}{lcr|lcr}
       & Assets & & Liabilities & \\
       \hline
       Cash in Vault & & 2,000 & Demand deposits & & 120,000\\
       Reserves at the Fed & & 33,000 & Other liabilities & & 10,000\\
       Business \& consumer loans & & 55,000 & Net worth & & 20,000\\
       US government securities & & 30,000 & & & \\
       Other Assests & & 30,000 & & & \\
       \hline
       & & 150,000 & & 150,000 \\
     \end{tabular}
   \end{center}
      
   \begin{parts}
     \part[1] The required reserves of First Baltimore are 
     \begin{solution}
       $$\$24,000$$
     \end{solution}

     \part[2]How much could the bank loan out?
     \begin{solution}
       Excess reserves are total reserves - required reserves. Thus
       for this bank it is $\$35,000 - \$24,000 = \$11,000$. This is
       also the amount the bank can use to make loans.
     \end{solution}

     \part[1] On the federal reserve balance sheet the First Baltimore
     bank will appear as an asset or liability?
     \begin{solution}
       liability.
     \end{solution}

     \part[3] As you examine the balance sheet, do you think assets
     have been chosen to reflect portfolio decisions aimed at
     maximizing profits? why or why not?
     \begin{solution}
       No , the bank is holding a lot of excess and therefore missing
       out on potentially profitable investment/loan
       opportunities. Also it is holding a large percentage of its
       assets in government securities; these are very low risk but
       low returns. 
     \end{solution}

     \part Starting sometime spring 1989 the Fed decided, the Fed
     decided that interst rates were too high and took steps to drive
     them down - a process that continued for more than three years.
     \begin{subparts}
       \subpart[1] Name one tool that the Fed could use to push down
       interest rates.
       \begin{solution}
         The Fed could purchase Treasure bills/bonds. Alternatively it
         could reduce the discount rate or lower required reserve
         ratios, though the latter is hardly ever used.
       \end{solution}
       \subpart[3] Carefully explain how the tool you selected could
       be used to reduce interest rates.
       \begin{solution}
         If the Fed bought T-bills, it would pay for the securities by
         issuing more reserves to, say, a bank from which it bought
         the T-bills. If the bank previously had no excess reserves,
         it would now find that it had excess reserves. It could loan
         these out, starting a process by which the initial increase
         in reserves could lead to a multiple expansion of the money
         supply via the banking system as a whole. In essence the
         money supply curve shifts right, this would reduce the
         equilibrium interest rate.
       \end{solution}
       \subpart[2] Explain the impact of lower interest rates on
       investment and aggregate output.
       \begin{solution}
         Lower interest rates will lower borrowing costs. In turn
         demand for investment and consumer durables should
         increase. Thus aggregate expenditures will rise, causing a
         multiple expansion in aggregate output.
       \end{solution}
     \end{subparts}
   \end{parts}

\end{questions}

\end{document}
