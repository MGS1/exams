% !TEX TS-program = pdflatex
% !TEX encoding = UTF-8 Unicode

\documentclass[12pt,addpoints]{exam} % use larger type; default would be 10pt
\usepackage[utf8]{inputenc} % set input encoding (not needed with XeLaTeX)
\usepackage{listings}
\usepackage{amssymb} % for \nmid


%%% Examples of Article customizations
% These packages are optional, depending whether you want the features they provide.
% See the LaTeX Companion or other references for full information.

%%% PAGE DIMENSIONS
\usepackage{geometry} % to change the page dimensions
\geometry{a4paper} % or letterpaper (US) or a5paper or....
% \geometry{margins=2in} % for example, change the margins to 2 inches all round
% \geometry{landscape} % set up the page for landscape
%   read geometry.pdf for detailed page layout information


% \usepackage[parfill]{parskip} % Activate to begin paragraphs with an empty line rather than an indent

%%% PACKAGES
\usepackage{booktabs} % for much better looking tables
\usepackage{array} % for better arrays (eg matrices) in maths
\usepackage{paralist} % very flexible & customisable lists (eg. enumerate/itemize, etc.)
\usepackage{verbatim} % adds environment for commenting out blocks of text & for better verbatim
\usepackage{subfig} % make it possible to include more than one captioned figure/table in a single float
\usepackage{amsthm} % for proof environment
% These packages are all incorporated in the memoir class to one degree or another...
\usepackage{amsfonts} % allows use of mathbb 
\usepackage{amsmath}
\usepackage{multicol}
\usepackage{graphicx} % support the \includegraphics command and options


%%% HEADERS & FOOTERS
%\usepackage{fancyhdr} % This should be set AFTER setting up the page geometry
%\pagestyle{fancy} % options: empty , plain , fancy
%\renewcommand{\headrulewidth}{0pt} % customise the layout...
%\lhead{}\chead{}\rhead{}
%\lfoot{}\cfoot{\thepage}\rfoot{}

%%% SECTION TITLE APPEARANCE
\usepackage{sectsty}
\allsectionsfont{\sffamily\mdseries\upshape} % (See the fntguide.pdf for font help)
% (This matches ConTeXt defaults)

%%% ToC (table of contents) APPEARANCE
\usepackage[nottoc,notlof,notlot]{tocbibind} % Put the bibliography in the ToC
\usepackage[titles,subfigure]{tocloft} % Alter the style of the Table of Contents
\renewcommand{\cftsecfont}{\rmfamily\mdseries\upshape}
\renewcommand{\cftsecpagefont}{\rmfamily\mdseries\upshape} % No bold!
\newcommand{\Dgrad}[1]{\textsf{grad}\,#1\;}
\newcommand{\del}[1] {\mathrm{d}#1\ }
\newcommand{\Q}{\mathbb{Q}}
\newcommand{\F}{\mathbb{F}}
\newcommand{\C}{\mathbb{C}}
\newcommand{\Z}{\mathbb{Z}}
\newcommand{\R}{\mathbb{R}}
\newcommand{\N}{\mathbb{N}}
\newcommand{\PP}{\mathbb{P}}
\newcommand{\arcsec}{\mbox{arcsec}}
\newcommand{\arccsc}{\mbox{arccsc}}
\newcommand{\arccot}{\mbox{arccot}}
\newcommand{\csch}{\mbox{csch}}
\newcommand{\sech}{\mbox{sech}}

%%% NEW ENVIRONMENTS
\newtheorem{theorem}{Theorem}
\newtheorem{definition}[theorem]{Definition}
\newtheorem{example}[theorem]{Example}
\newtheorem{proposition}[theorem]{Proposition}
\newtheorem{corollary}[theorem]{Corollary}
\newtheorem{lemma}[theorem]{Lemma}

%%% END Article customizations

%%% The "real" document content comes below...


\usepackage{color}
\lstset{numbers=left}
\begin{document}

\printanswers

\title{Physics C: Part I}
\maketitle



\begin{center}
    This exam has \numquestions\ questions for a total of \numpoints\
    points.You have {\bf 2 Hours} to complete this exam.
\end{center}

% Hwk#  Wolfson/Pasachoff    Halliday/Resnick
% 1                    1 - 4                   1 - 4
% 2                       5                       5
% 3                       6, 7.1-7.2         6, 7.1-7.3
% 4                       fin 7, 8              fin 7, 8
% 5                       9                       15
% 6                       10 & 11             9 & 10
% 7                       12                     11
% 8                       13-14               12-13
% 9a                     15, 16                14, 17                        
% 9b                     17, 19                18, 19
% 10                     20, 21                20, 21

% 10 problem sets, so 12 points per set.


\begin{questions}

  \section{Motion in a plane}

  \question (\totalpoints\ points) A soccer player kicks a ball at
  an angle of $37^\circ$ above the horizontal with an initial speed
  of 20.0 m/s.
  \begin{parts}
    \part[4] Find the time $t$ at which the ball reaches the highest
    point of its trajectory.
    \begin{solution}
      $$v_y(t) = 20.0 \sin(37 t) - 9.8t$$
      $$t = 1.2\mbox{s}$$
    \end{solution}
    
    \part[2] What is the range of the ball assuming the ground is
    level.
    \begin{solution}
      $$R = \frac{v_0^2}{g} \sin  2\theta = 39\mbox{m}$$
    \end{solution}
    
    \part[4] Ignoring air resistance, what is the ball's vertical
    position when it is 10.0 m down range from its starting
    position?
    
    \begin{solution}
      $$x_x(t) = 10.0\mbox{m} = 20.0 \cos 37t$$
      $$t = .63\mbox{s}$$
      $$x_y(.63) = 5.6$$
    \end{solution}
    
  \end{parts}
  
  \question[1] Can an object have a zero velocity and still be
  accelerating?
  
  \begin{solution}
    Yes, if you throw a ball up at the vertex.
  \end{solution}
  
  \question[1] Can an object be increasing in speed as its
  acceleration decreases? 
  \begin{solution}
    Yes.
  \end{solution}
  
  \section{Force}
  
  \question[2] A 5 kg concrete block is lowered with a downward
  acceleration of 2.8 m/s$^2$ by means of a rope. The force of the
  block on the rope is:
  \begin{oneparchoices}
    \choice 14 N up
    \choice 14 N down
    \choice 35 N up
    \CorrectChoice 35 N down
    \choice 49 N up
  \end{oneparchoices}
  
  \question[2] The ``reaction'' force does not cancel the ``action''
  force beacuse:
  \begin{oneparchoices}
    \choice action force is greater
    \CorrectChoice they act on different bodies
    \choice they are in the same direction
    \choice the reaction force exists only after the action force is
    removed
    \choice reaction force is greater than the action force
  \end{oneparchoices}
  
  \question[2] A block slides down a frictionless plane which makes
  an angle of $30^\circ$ with the horizontal. The acceleration of
  the block (in cm/s$^2$) is:
  \begin{oneparchoices}
    \choice 980
    \choice 566
    \choice 849
    \choice zero
    \CorrectChoice 490
  \end{oneparchoices}
  
  \question[2] A 1000 kg elevator is rising and its speed is
  increasing at 3 m/s$^2$. The tension in the elevator cable is:
  \begin{oneparchoices}
    \choice 6800 N
    \choice 1000 N
    \choice 3000 N
    \choice 9800 N
    \CorrectChoice 12800 N
  \end{oneparchoices}
  
  \section{Force and Friction}
  
  \question[2] A box rests on a rough board 10 meters long. When one
  end of the board is slowly raised to a height of 6 meters above the
  other end, the box begins to slide. The coefficient of static
  friction is:
  \begin{oneparchoices}
    \choice 0.8
    \choice 0.25
    \choice 0.4
    \choice 0.6
    \CorrectChoice 0.75
  \end{oneparchoices}

  \question[2] A horizontal force of 12 N pushes a 0.50 kg block
  against a vertical wall. The block is initially at rest. If $\mu_s =
  0.6$ and $\mu_k = 0.80$, the acceleration of the block in m/s$^2$
  is:
  \begin{oneparchoices}
    \CorrectChoice 0
    \choice 9.4
    \choice 9.8
    \choice 14.4
    \choice 19.2
  \end{oneparchoices}

  \question[2] Why do raindrops fall with constant speed during the
  later stages of their descent?
  \begin{oneparchoices}
    \choice the gravitational force is the same for all drops
    \CorrectChoice air resistance just balances the force of gravity
    \choice the drops all fall from the same height
    \choice the force of gravity is negligible for objects as small as
    raindrops
    \choice gravity cannot increase the speed of a falling object to
    more than 32 ft/s.
  \end{oneparchoices}

  \question (\totalpoints\ points) A man drags a 80 kg crate across a
  floor by pulling a rope that makes an angle of 15$^\circ$ above the
  horizontal.
  \begin{parts}
    \part[5] If the coefficient of static friction is 0.50 what force
    must the man exert to start the crate moving.
    \begin{solution}
      $$F = f/\cos \theta = \frac{0.50 N}{\cos \theta} = \frac{0.50
        (mg - F_m \sin \theta)}{\cos \theta}$$
      $$F + \frac{0.5F \sin \theta}{\cos \theta} = \frac{0.5 mg}{\cos
        \theta}$$
      $$F = \frac{0.5 mg}{\cos \theta (1 + 0.5 \tan \theta)}$$
      $$= 357.89 N$$
    \end{solution}
    \part[5] If the coefficient of kinetic friction is 0.35 what is
    the acceleration of the crate just after it starts moving while
    the man is still exerting the force of part (a)?
    \begin{solution}
      $$a = 1.3 m/s^2$$
    \end{solution}
  \end{parts}
  
  \section{Work and Energy}

  \question (\totalpoints\ points) A 10.0 kg block slides 3.0 m down a
  plane that is inclined at an angle of $20^\circ$ to the
  horizontal. The coefficient of kinetic friction between the block
  and the plane is 0.10.
  \begin{parts}
    \part[1] Calculate the work done by the gravitational force acting
    on the block.

    \part[1] Calculate the work done by the force of kinetic friction
    acting on the block.

    \part[1] Calculate the work done by the normal force acting on the
    block.

    \part[1] Calculate the total (net) work done on the block.

    \part[3] What is the speed of the block as it passes the end of
    the 3.0 m section assuming it was moving at 1.0 m/s as it passed
    the beginning of the 3.0 m section? {\bf Use the work-energy
      theorem.}

    \part[1] What instantaneous power is being delivered to the block
    as it passes the end of the 3.0 m setion?

    \part[1] What was the average power delivered to the block while
    it moved through the 3.0 m section?

  \end{parts}

\end{questions}


\end{document}
